% Author: Pranshu Bansal
% Email: pranshu@berkeley.edu
\learning{
Students should be comfortable working with basic vector operations (such as addition) matrix vector multiplications.
}

Consider the following:

$$\vec{v}_1 = \begin{bmatrix} 4 \\ 7 \\ -5 \end{bmatrix}$$
$$\vec{v}_2 = \begin{bmatrix} 1 \\ 3 \\ -1 \end{bmatrix}$$
$$\vec{v}_3 = \begin{bmatrix} 2 \\ 4 \end{bmatrix}$$
$$A = \begin{bmatrix} 1 & 3 \\7 & 9 \end{bmatrix}$$
$$B = \begin{bmatrix} 5 & 1 \\3 & 6 \end{bmatrix}$$


\begin{enumerate}

\item What is the transpose of $\vec{v}_1$?

\sol{
  $$\vec{v}_1' = \begin{bmatrix} 4 & 7 & -5 \end{bmatrix}$$
}

\item What is $\vec{v}_1 + \vec{v}_2$?

\sol{
  $$\vec{v}_1 + \vec{v}_2 = \begin{bmatrix} 4 + 1 \\ 7 + 3 \\ -5 -1 \end{bmatrix}$$
  $$\vec{v}_1 + \vec{v}_2 = \begin{bmatrix} 5 \\ 10 \\ -6 \end{bmatrix}$$
}

\item What is $2\vec{v}_1 - 3\vec{v}_2$?

\sol{
  $$2\vec{v}_1 -3\vec{v}_2 = \begin{bmatrix} 2*4 - 3*1 \\ 2*7 - 3*3 \\ 2*(-5) -3*(-1) \end{bmatrix}$$
  $$2\vec{v}_1 -3\vec{v}_2 = \begin{bmatrix} 5 \\ 5 \\ -7 \end{bmatrix}$$
}

\item What is $\vec{v}_1 \cdot \vec{v}_2$ (dot product)?

\sol{
  $$\vec{v}_1 \cdot \vec{v}_2 = \vec{v}_1^T \vec{v}_2$$
  $$\vec{v}_1 \cdot \vec{v}_2 = \begin{bmatrix} 1 & 3 & -1 \end{bmatrix} \begin{bmatrix} 4 \\ 7 \\ -5 \end{bmatrix}$$

  $$\vec{v}_1 \cdot \vec{v}_2 = 4*1 + 7*3 + (-5)*(-1)$$
  $$\vec{v}_1 \cdot \vec{v}_2 = 30$$

  Inner product of two vectors having the same dimensions is the sum of products of corresponding terms in the vectors. Inner products are undefined for vectors of different dimensions.
}

\item What is $A \vec{v}_3$?

\note{
 Make sure students internalize the structure of matrix vector multiplication (maybe replace the matrix with row vector variables to show where things end up)

 Also, students should take note of the orientation of the vectors, whether they're row vectors or column vectors. A row vector multiplied by a column vector would equal a scalar, but a column vector multiplied by a row vector would equal a matrix. 
}


\sol{
  $$A \vec{v}_3 = \begin{bmatrix} 1 & 3\\7 & 9 \end{bmatrix} \begin{bmatrix} 2 \\ 4 \end{bmatrix}$$

  $$A \vec{v}_1 = \begin{bmatrix} 1*2 + 3*4 \\7*2 + 9*4 \end{bmatrix}$$

  Note: Matrix vector multiplication is just stacked vector vector dot products. The first row of the product is the same as the answer to the last problem.

  $$A \vec{v}_1 = \begin{bmatrix} 14 \\ 50 \end{bmatrix}$$

}

\item What is $AB$?

\sol{
  $$AB = \begin{bmatrix} 1 & 3\\7 & 9 \end{bmatrix} \begin{bmatrix} 5 & 1 \\3 & 6 \end{bmatrix}$$

  $$AB = \begin{bmatrix} (1*5 + 3*3) & (1*1 + 3*6)\\(7*5 + 9*3) & (7*1 + 9*6) \end{bmatrix}$$
  $$AB = \begin{bmatrix} 14 & 19 \\ 62 & 61 \end{bmatrix}$$

}

\end{enumerate}
\newpage