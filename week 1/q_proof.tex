\learning{
    Students should be comfortable working with proofs in linear algebra that stems from definitions and baisc properties.
}
\\ \\
Let $A$ be an $m \times n$ matrix. Show that the following 4 statements about $A$ are all \textbf{logically equivalent}. That is, for a particular $A$, either these statements are all true or they are all false.
\\
\statement{
    \begin{enumerate}
        \item For each $\vec{b}$ in $\mathbb{R}^m$, the equation $A\vec{x} = \vec{b}$ has a solution.
        \item Each $\vec{b}$ in $\mathbb{R}^m$ is a linear combination of the columns of $A$.
        \item The columns of $A$ span $\mathbb{R}^m$,
        \item $A$ has a pivot position in every row.
    \end{enumerate}
}
\\ \\
\textbf{\textit{Hint}}: Specifically, \textbf{show that the following statements are equivalent}:
\\ \\
\statement{
    \begin{itemize}
        \item Statement 1 is equivalent to statement 2
        \item Statement 2 is equivalent to statement 3
        \item Statement 1 is equivalent to statement 3
        \item Statement 1 is equivalent to statement 4
    \end{itemize}
}
\\ \\
\textbf{\textit{Another Hint:}} In general, to show two statements are equivalent, we can take either of the approaches below:
\begin{itemize}
    \item Show that the statements carry the same meaning by transforming the definitions/properties in one of the statements into those expressed in the other.
    \item Show that:
    \begin{enumerate}
        \item If one statement is true, the other one must also be true.
        \item If one statement is false, the other one must also be false. 
    \end{enumerate}
    \textbf{It is important that both cases be justified}.
\end{itemize}
\sol{\\
    Let's first show \textbf{tatement 1 is equivalent to statement 2}. \\ \\
    By definitions of the matrix vector product $A\vec{x}$ and span of a set of vectors in $\mathbb{R}^m$, define: 
    $$A = \begin{bmatrix}
        \uparrow & \uparrow & \hdots & \uparrow \\
        \vec{a}_1 & \vec{a}_2 & \hdots & \vec{a}_n \\
        \downarrow & \downarrow & \downarrow & \downarrow
    \end{bmatrix}, \:\: \vec{x} = \begin{bmatrix}
        x_1 \\
        x_2 \\
        \hdots \\
        x_n
    \end{bmatrix}$$
    If the equation $A\vec{x} = \vec{b}$ has a solution, that means \textbf{the equality holds}. Substituting the more specific expressions we defined above (replace $A$ with its column vectors $\vec{a}_1$ through $\vec{a}_n$, and replace $\vec{x}$ with its entries $x_1$ through $x_n$), we have:
    $$
    \begin{bmatrix}
        \uparrow & \uparrow & \hdots & \uparrow \\
        \vec{a}_1 & \vec{a}_2 & \hdots & \vec{a}_n \\
        \downarrow & \downarrow & \downarrow & \downarrow
    \end{bmatrix}\begin{bmatrix}
        x_1 \\
        x_2 \\
        \hdots \\
        x_n
    \end{bmatrix} = \vec{b}
    $$
    $$
    x_1 \vec{a}_1 + x_2 \vec{a}_2 + \hdots + x_n \vec{a}_n = \vec{b}
    $$
    Since $
    x_1 \vec{a}_1 + x_2 \vec{a}_2 + \hdots + x_n \vec{a}_n
    $ is a linear combination of the column vectors of $A$ by definition, from this equation, we have shown that $\vec{b}$ is expressed as a linear combination of the columns of $A$ if $A\vec{x} = \vec{b}$ has a solution. $\square$
    \\ \\
    Next, let's show \textbf{statement 2 is equivalent to statement 3}. \\ \\
    Since we are looking at all $\vec{b}$ in $\mathbb{R}^m$, and a linear combination of the columns of $A$ has the form $c_1 \vec{a}_1 + c_2 \vec{a}_2 + \hdots + c_n \vec{a}_n$, if every such vector $\vec{b}$ can be expressed as $c_1 \vec{a}_1 + c_2 \vec{a}_2 + \hdots + c_n \vec{a}_n$, that means the linear combination of the columns vectors of $A$ can reach any vector in $\mathbb{R}^m$. \\ \\
    Equivalently, this implies the columns of $A$ span $\mathbb{R}^m$. $\square$ \\ \\
    Now, to show \textbf{statement 1 is equivalent to statement 3},
    notice we have shown that \textbf{statement 1 is equivalent to statement 2} and \textbf{statement 2 is equivalent to statement 3}, and we know that logical equivalences (just like equalities) are transitive (i.e. If $a = b$, $b = c$, then $a=c$), we conclude that \textbf{statement 1 must also be equivalent to statement 3} as well. $\square$
    \\ \\
    Finally, let's show \textbf{statement 1 is equivalent to statement 4}. \\ \\
    Let $U$ be an echelon form of $A$. Given the vector $\vec{b}$ in $\mathbb{R}^m$, we can row reduce the augmented matrix $\begin{bmatrix}
        A & \vec{b}
    \end{bmatrix}$ to an augmented matrix $\begin{bmatrix}
        U & \vec{d}
    \end{bmatrix}$ for some different vector $\vec{d}$ in $\mathbb{R}^m$:
    $$\begin{bmatrix}
        A & \vec{b}
    \end{bmatrix} \sim \cdots \sim \begin{bmatrix}
        U & \vec{d}
    \end{bmatrix}.$$
    If statement 4 is indeed true, then each row of $U$ must contain a pivot position and \textbf{there can be no pivot in the augmented column}. So $A\vec{x} = \vec{b}$ has a solution for any $\vec{b}$, and statement 1 must also be true. \\ \\
    On the other hand, if statement 4 is false, the last row of $U$ will be all zeros. Let $\vec{d}$ be any vector with a 1 in its last entry, then $\begin{bmatrix}
        U & \vec{d}
    \end{bmatrix}$ represents an inconsistent system. Since \textbf{the row operations we use when reducing a matrix are reversible}, this means $\begin{bmatrix}
        U & \vec{d}
    \end{bmatrix}$ can be transformed back into the form $\begin{bmatrix}
        A & \vec{b}
    \end{bmatrix}$. Hence, the new system $A\vec{x} = \vec{b}$ is also inconsistent, and statement 1 will be false as well. $\square$ \\ \\
    Hence, we have shown that all of the 4 statements are logically equivalent. $\square$
}