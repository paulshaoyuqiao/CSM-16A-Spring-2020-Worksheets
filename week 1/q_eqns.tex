% Author: Mudit Gupta, Elena Herbold, Paul Shao, Pranshu Bansal
% Email: mudit+csm16a@berkeley.edu
%       eherbold@berkeley.edu
% .     paulshaoyuqiao1@berkeley.edu
%       pranshu@berkeley.edu
\learning{
Students should be comfortable solving a three-variable system of equations using GE with the forward/backward elimination method. Additionally, they should know how to convert a solution with a free variable from equations describing the solution set into vector notation. \\
Description: Simple mechanical gaussian elimination problem + some insight about span and free variables}


\begin{enumerate}

\item Consider the following set of linear equations:

$$1x - 3y +1z = 4$$
$$2x -8y +8z = -2$$
$$-6x+3y-15z = 9$$

Place these equations into a matrix $A$, and row reduce $A$ to solve the equations.

\note{
  This semester, 16A is making the distinction between a matrix in row-echelon form (REF) and reduced row-echelon form (RREF). Per Note 1, an REF matrix looks something like
  $$\left[\begin{array}{cccc|c}  
  1 & * & * & * & * \\  
  0 & 1 & * & * & * \\
  0 & 0 & 0 & 1 & * \\
  0 & 0 & 0 & 0 & 0
  \end{array}\right]$$ whereas an RREF matrix looks like
  $$\left[\begin{array}{cccc|c}  
  1 & 0 & * & 0 & * \\  
  0 & 1 & * & 0 & * \\
  0 & 0 & 0 & 1 & * \\
  0 & 0 & 0 & 0 & 0
  \end{array}\right]$$ the difference being that in an RREF matrix has only 1's or 0's in a column with a pivot.
}

\sol{
  $$A = \begin{bmatrix} 1 & -3 & 1 \\ 2 & -8 & 8 \\ -6 & 3 & -15 \end{bmatrix} \begin{bmatrix} x \\ y \\ z \end{bmatrix} = \begin{bmatrix} 4 \\ -2 \\ 9 \end{bmatrix}$$

  $$R_2 = R_2 - 2*R_1$$
  $$R_3 = R_3 + 6*R_1$$

  $$A = \begin{bmatrix} 1 & -3 & 1 \\ 0 & -2 & 6 \\ 0 & -15 & -9 \end{bmatrix} \begin{bmatrix} x \\ y \\ z \end{bmatrix} = \begin{bmatrix} 4 \\ -10 \\ 33 \end{bmatrix}$$

  $$R_2 = R_2/2$$
  $$R_3 = R_3/3$$
  
  $$A = \begin{bmatrix} 1 & -3 & 1 \\ 0 & -1 & 3 \\ 0 & -5 & -3 \end{bmatrix} \begin{bmatrix} x \\ y \\ z \end{bmatrix} = \begin{bmatrix} 4 \\ -5 \\ 11 \end{bmatrix}$$

  $$R_3 = R_3 - 5*R_2$$

  $$A = \begin{bmatrix} 1 & -3 & 1 \\ 0 & -1 & 3 \\ 0 & 0 & -18 \end{bmatrix} \begin{bmatrix} x \\ y \\ z \end{bmatrix} = \begin{bmatrix} 4 \\ -5 \\ 36 \end{bmatrix}$$

  $$R_2 = R_2 * -1$$
  $$R_3 = R_3 / -18$$

  $$A = \begin{bmatrix} 1 & -3 & 1 \\ 0 & 1 & -3 \\ 0 & 0 & 1 \end{bmatrix} \begin{bmatrix} x \\ y \\ z \end{bmatrix} = \begin{bmatrix} 4 \\ 5 \\ -2 \end{bmatrix}$$

  This form of the matrix is called the row echelon form or the REF.

  $$R_2 = R_2 + 3*R_3$$

  $$A = \begin{bmatrix} 1 & -3 & 1 \\ 0 & 1 & 0 \\ 0 & 0 & 1 \end{bmatrix} \begin{bmatrix} x \\ y \\ z \end{bmatrix} = \begin{bmatrix} 4 \\ -1 \\ -2 \end{bmatrix}$$

  $$R_1 = R_1 + 3*R_2 - R_3$$

  $$A = \begin{bmatrix} 1 & 0 & 0 \\ 0 & 1 & 0 \\ 0 & 0 & 1 \end{bmatrix} \begin{bmatrix} x \\ y \\ z \end{bmatrix} = \begin{bmatrix} 3 \\ -1 \\ -2 \end{bmatrix}$$

  Now, we have reduced the matrix to the reduced row echelon form or the RREF.

  $$z = -2$$
  $$y = -1$$
  $$x = 3$$
}

\item Consider another set of linear equations:

$$2x + 3y +5z = 0$$
$$-1x -4y -10z = 0$$
$$x-2y-8z = 0$$

Place these equations into a matrix $A$, and row reduce $A$.

\note{
  Note to mentors: When you do Gaussian Elimination, start by making $a_{2,1} = 0$ using some multiple of $a_{1,1}$. Next, make $a_{3,1} = 0$ using some multiple of $a_{1,1}$. Next, make $a_{3,2} = 0$ by using some multiple of $a_{2,2}$. In this last step, when you use row 2’s pivot to subtract out row 3, the first element of row 3 will not be affected (it will remain 0). This is because in the previous steps, we got rid of the first element of row 2 as well. This is what I like to call the zig zag method of doing Gaussian Elimination. (Elena: I call this the ’staircase’, and this is the Gaussian Elimination method 16A currently teaches. I think it is officially called forward/backward elimination.) Start at the top left, move down the column. Then start again at the top of the second column and move down. \\
  Another way of thinking about this process is that when you go forward, you’re putting the matrix into REF form, and going backwards puts it into RREF form.
}

\sol{
  $$A = \begin{bmatrix} 
  2 & 3 & 5 \\
  -1 & -4 & -10 \\
  1 & -2 & -8 
  \end{bmatrix}$$

  $$R_2 = R_2 + \frac{1}{2}R_1$$
  $$R_3 = R_3 - \frac{1}{2}R_1$$

  $$A = \begin{bmatrix} 
  2 & 3 & 5 \\
  0 & -2.5 & -7.5 \\
  0 & -3.5 & -10.5 
  \end{bmatrix} \\$$

  Remember that we can only do row operations without caring about the RHS because the RHS is all zeroes. Hence, any linear row operations won't affect the RHS i.e. it will remain the zero vector.

  Make the numbers nicer by dividing row 2 by -2.5, and multiplying row 3 by -2. This is always a good thing to do if you realize your numbers are getting messy! (Also, feel free to keep all the numbers as non-fractional values by finding the least common multiple of the two numbers you are trying to cancel out.) 

  $$R_2 = \frac{1}{-2.5}R_2$$ 
  $$R_3 = -2R_3$$

  $$A = \begin{bmatrix} 
  2 & 3 & 5 \\
  0 & 1 & 3 \\
  0 & 7 & 21 
  \end{bmatrix} $$

  $$R_3 = R_3 - 7R_2$$

  $$A = \begin{bmatrix} 
  2 & 3 & 5 \\
  0 & 1 & 3 \\
  0 & 0 & 0 
  \end{bmatrix}$$

  $$R_1 = R_1 / 2$$

  $$A = \begin{bmatrix} 
  1 & 1.5 & 2.5 \\
  0 & 1 & 3 \\
  0 & 0 & 0 
  \end{bmatrix}$$

  This is the REF of the equation matrix.

  $$R_1 = R_1 - 1.5*R_2$$
  
  $$A = \begin{bmatrix} 
  1 & 0 & -2 \\
  0 & 1 & 3 \\
  0 & 0 & 0 
  \end{bmatrix}$$

  This is the RREF of the equation matrix.


}

\item
Convert the row reduced matrix back into equation form. 

\note{
  Note that although the equations have infinite solutions, there are still some constraints on $x$, $y$, and $z$: for example, choosing $x=y=z=1$ wouldn’t work, since plugging it into the first equation would give us $2 \cdot 1 + 3 \cdot 1 + 5 \cdot 1 = 10 \ne 0$. This is shown concretely later in the problem, but be sure to keep this in mind in case a student says something along the lines of “any numbers work.”
}

\sol{
  \begin{bmatrix} 
  1 & 0 & -2 \\
  0 & 1 & 3 \\
  0 & 0 & 0 
  \end{bmatrix}$$ 
  \begin{bmatrix} x \\ y \\ z \end{bmatrix} = \begin{bmatrix} 0 \\ 0 \\ 0 \end{bmatrix}$$
  $$1x +0y - 2z = 0$$
  $$0x + 1y + 3z = 0$$
  $$ 0x + 0y + 0z = 0$$
}

\item
Intuitively, what does the last equation from the previous part tell us?

\note{
  It tells us that there are infinite solutions to the equations. $0x + 0y + 0z = 0$ is satisfied by $\textbf{any}$ $x, y, z$. 
}

\sol {
  If students are confused at this point about why we can infer this, their confusion is well justified. Suppose that there were \textbf{4} equations in 3 variables -- 3 of them were linearly independent, and the fourth one was $0x + 0y + 0z = 0$, then the system still has just 1 solution. The last equation is never \textit{used} in some sense. Feel free to talk about this with students. Present it as: what if you had 4 equations, you wrote them in matrix form, got pivots in all rows except for one where you got a row of all 0s -- are there still infinite solutions? The answer is no. 
}

\item How many pivots are there in the row reduced matrix? What are the free variables?

\note{It is important to emphasize to the students that for this particular matrix, either $y$ or $z$ can be a free variable. By Gaussian Elimination's convention, however, we always choose in the priority of right to left (in this case, we choose $z$ as our free variable)}

\sol{
    There are 2 pivots in this row reduced matrix, and the corresponding pivot columns (following Gaussian Elimination's convention) are column 1 and 2. There are no more pivots since the third row are all zeros, and we require a non-zero element at the position of the third column (following column 2) and the third row for there to be one more pivot.
    
    The free variables can be $y$ or $z$ in this case, but we choose $z$ as our free variable by Gaussian Elimination's convention. 
}

\item What is the dimension of the span of all the column vectors in $A$?

\note{
    If the students are getting overwhelmed by all the technical terminologies in this question, try to break the question down term by term:
    
    dimension: the number of entries in the vector that can take on infinitely many values (for example, the dimension of a 1D number line is 1, the dimension of a 2D grid is 2, and the dimension of a 3D space is 3).
    
    span: the set of linear combinations of a set of vectors
    
    column vectors: we treat each column in the matrix $A$ as a vector (call it column vector)
    
    It is also helpful to remind them think about the row reduced matrix we have in the end and the number of pivots we have.
}

\sol{
    As we can see, in the row reduced form of $A$, since the third row are all zeros, and there are only 2 pivots with $z$ as the free variable, the dimension of the span of all the column vectors in $A$ is equal to 2 (number of pivots).
    
    Alternatively, we can follow the definition of a span and algebraically write out the linear combinations of all the column vectors in the row reduced form of $A$, and we can see that the third entry in the resulting linear combination will always be 0 (since all 3 column vectors have 0's in their third entries), hence there are only 2 dimensions (entries) in the resulting vectors whose values we have control over. 
}

\item {Now that we've established that this system has infinite solutions, does every possible combination of $x, y, z \in \mathbb{R}$ solve these equations}

\note{This is supposed to be a quick part. Explain that the existence of infinite solutions doesn't mean that all possible combinations work.}

\sol{No. $x=1, y=1, z=1$ doesn't work, for instance.}

\item
What is the general form (in the form of a constant vector multiplied by a variable $t$) of the infinite solutions to the system? 

\note{
  Explain why $z$ is the free variable. (Because it is the one that doesn't have a pivot in the corresponding column). Also explain what "general form" means if students are confused.

  Students might be confused by the idea of a “general form” for the solution: try to convey that the solution is an equation that describes all possible solutions. If we assign some fixed value for $z$, then we can solve for a single pair of values for $x, y$. In the general solution, we replace $z$ with a parameter variable $t$. The equations in the solution come from setting $z=t$ (our free variable), and plugging these back into the equations from our row-reduced matrix.
}

\sol{
  $z$ is a free variable. If $z = t$, then
  $$y = -3z = -3t$$
  $$x - 2z = 0 \implies x = 2z = 2t$$

  The general solution is then $t \begin{bmatrix} 2 \\ -3 \\ 1 \end{bmatrix}$. What this means is that any multiple of the vector $\begin{bmatrix} 2 \\ -3 \\ 1 \end{bmatrix}$ will satisfy the equations. Try it!
}


\end{enumerate}
\newpage