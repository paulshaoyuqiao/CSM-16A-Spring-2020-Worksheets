% Author: Pranshu Bansal
% Email: pranshu@berkeley.edu
\learning{
Proof on spans and linear dependence/independence
} \\
    Let {$\vec{v}_1,\vec{v}_2, \dots, \vec{v}_n$} be a set of vectors V. Prove that if the set of vectors is linearly dependent, then at least one vector can be deleted from the set without diminishing its span.

\note{
    Make sure to emphasize that v is just some arbitrary vector in the span of the set, while v1 is a linearly dependent vector in the set. This will make it more clear when the substitution (for v1) step happens. 
}

\sol{
  The general form of a vector $\vec{v}$ in the span of {$\vec{v}_1,\vec{v}_2, \dots, \vec{v}_n$} is $c_1\vec{v}_1 + c_2\vec{v}_2 + \dots + c_n\vec{v}_n$.

  Without loss of generality, let us assume $\vec{v}_1$ can be written as the linear combination of the remaining vectors as $a_2\vec{v}_2 + a_3\vec{v}_3 + \dots + a_n\vec{v}_n$.

  $$\vec{v} = c_1\vec{v}_1 + c_2\vec{v}_2 + \dots + c_n\vec{v}_n$$

  If we substitute $\vec{v}_1$ for the above value ($a_2\vec{v}_2 + a_3\vec{v}_3 + \dots + a_n\vec{v}_n$) in the general form of $\vec{v}$, we get: 

  $$\vec{v} = c_1*(a_2\vec{v}_2 + a_3\vec{v}_3 + \dots + a_n\vec{v}_n) + c_2\vec{v}_2 + \dots + c_n\vec{v}_n$$

  $$\vec{v} = c_1a_2\vec{v}_2 + c_1a_3\vec{v}_3 + \dots + c_1a_n\vec{v}_n + c_2\vec{v}_2 + \dots + c_n\vec{v}_n$$

  $$\vec{v} = (c_1a_2 + c_2)\vec{v}_2 + (c_1a_3 + c_3)\vec{v}_3 + \dots + (c_1a_n + c_n)\vec{v}_n$$

  We can see that any vector we could represent as a linear combination of the vectors in V can be represented without using $\vec{v}_1$ using the new parameters we got in the above equation.

  Hence, if the set of vectors is linearly dependent, then at least one vector can be deleted from the set without diminishing its span.

}

\newpage