% Author: Zachary Golan-Strieb, Paul Shao
% Email: zacharyjgs@berkeley.edu, paulshaoyuqiao1@berkeley.edu
In this question, we will learn about the underlying transformations that allow us to find the inverse of a given matrix by exploring how matrices can be used to represent different types of row operations.

\learning{Students should understand matrix multiplication, linear transformations, Gaussian elimination}


\begin{enumerate}

\item What matrix $B$ can we multiply by a $3 \times 3$ matrix $M$ to get a new matrix $M'$ that is the same as $M$ but with row $2$ scaled by $1/5$?

\note{
Go through the steps of matrix multiplication to illustrate how putting a scalar $a$ in $i$th column of $A$ grabs elements from the $i$th row of $M$ multiplied by $a$.
}

\sol{
Let $M = 
\begin{bmatrix} 
\vec{m_1}^T \\
\vec{m_2}^T \\
\vec{m_3}^T
\end{bmatrix}$
and $B =
\begin{bmatrix} 
a_1 & 0 & 0 \\
0 & a_2 & 0 \\
0 & 0 & a_3
\end{bmatrix}$
then $BM =
\begin{bmatrix} 
a_1 \vec{m_1}^T \\
a_2 \vec{m_2}^T \\
a_3 \vec{m_3}^T
\end{bmatrix}$
so if we want to scale row $2$ by $1/5$ and leave the other rows unchanged, then we should set $a_1 = 1$, $a_2 = \frac{1}{5}$, and $a_3 = 1$. 
$$B = 
\begin{bmatrix} 
1 & 0 & 0 \\
0 & \frac{1}{5} & 0 \\
0 & 0 & 1 
\end{bmatrix}$$
}

\item What matrix $A$ can we left multiply by a $3 \times 3$ matrix $M$ to get a new matrix $M'$ that is the same as $M$ but with the row $1$ and row $3$ swapped?

\note{
Go through the steps of matrix multiplication to illustrate how putting a 1 in $i$th column of $A$ grabs elements from the $i$th row of $M$.
}

\sol{
Let $M = 
\begin{bmatrix} 
\vec{m_1}^T \\
\vec{m_2}^T \\
\vec{m_3}^T
\end{bmatrix}$
and $B =
\begin{bmatrix} 
a_{1,1} & a_{1,2} & a_{1,3} \\
a_{2,1} & a_{2,2} & a_{2,3} \\
a_{3,1} & a_{3,2} & a_{3,3}
\end{bmatrix}$
then $BM =
\begin{bmatrix} 
a_{1, 1} \vec{m_1}^T + a_{1, 2} \vec{m_2}^T + a_{1, 3} \vec{m_3}^T \\
a_{2, 1} \vec{m_1}^T + a_{2, 2} \vec{m_2}^T + a_{2, 3} \vec{m_3}^T \\
a_{3, 1} \vec{m_1}^T + a_{3, 2} \vec{m_2}^T + a_{3, 3} \vec{m_3}^T
\end{bmatrix}$
so if we want to swap row 1 and row 3 and leave row 2 unchanged, then we should make $a_{1, 3} = 1$ since it will put one of row 3 in row 1,  $a_{3, 1} = 1$ since it will put one of row 1 in row 3, $a_{2, 2} = 1$ to keep row 2 the same, and the remaining $a_{i, j} = 0$.

$$B =
\begin{bmatrix} 
0 & 0 & 1 \\
0 & 1 & 0 \\
1 & 0 & 0 
\end{bmatrix}$$
}

\item What matrix $C$ can we left multiply by a $3 \times 3$ matrix $M$ to get a new matrix $M'$ that is the same as $M$ but with the 3 times row $1$ added to the row $2$?

\note{
Go through the steps of matrix multiplication again to show how the values from row 1 and row 2 of $M$ are summed to create row 2 of $M'$.
}

\sol{
To get this matrix, you can use the method from the solution to part (b), but make $a_{2, 1} = 3$ to put $3$ times row 1 in row 2, $a_{1, 1} = a_{2, 2} = a_{3, 3} = 1$ to keep the original rows except for what we added to row 2, and the remaining $a_{i, j} = 0$.

$$B =
\begin{bmatrix} 
1 & 0 & 0 \\
3 & 1 & 0 \\
0 & 0 & 1 
\end{bmatrix}$$
}

\item What is the the multiplicative inverse of 2? What is the multiplicative identity? What is the additive inverse of 1? What is the additive identity? What is the identity in matrix/vector multiplication?

\note{
The point of this question is to get students to understand what an inverse and identity are. Multiplying something by or adding it to its inverse should give you the identity. Multiplying anything by or adding it to the identity should leave it unchanged. This is why multiplying by the identity matrix has no effect.
}

\sol{
The multiplicative inverse of $2$ is $\frac{1}{2}$. The multiplicative identity is $1$. The additive inverse of $1$ is $-1$. The additive identity is $0$. The identity for matrix multiplication is $\begin{bmatrix} 
1 & 0 & 0 \\
0 & 1 & 0 \\
0 & 0 & 1 
\end{bmatrix}$.
}

\item In what order should we apply the transformations described in parts (a), (b), and (c) to the matrix $M = 
\begin{bmatrix} 
0 & 0 & 1 \\
-15 & 5 & 0 \\
1 & 0 & 0 
\end{bmatrix}$ 
to get the identity matrix?

\note{
Make sure students understand that these would be the steps taken in Gaussian elimination.
}

\sol{
Swap row $1$ and row $3$, then scale row $2$ by $1/5$, then add 3 times row $1$ to row $2$.
}

\item Multiply the matrices for each transformation in the order determined in part (d). What happens when you multiply $M$ by this matrix? What is this matrix called?

\note{
This part of the question ties together the transformations from the earlier parts with the concept of matrix inverses. If you want, show how to find an inverse using Gaussian elimination and how that is the same as keeping track of each transformation as done above.
}

\sol{
$$
\begin{bmatrix} 
1 & 0 & 0 \\
3 & 1 & 0 \\
0 & 0 & 1 
\end{bmatrix}
\begin{bmatrix} 
1 & 0 & 0 \\
0 & 5 & 0 \\
0 & 0 & 1 
\end{bmatrix}
\begin{bmatrix} 
0 & 0 & 1 \\
0 & 1 & 0 \\
1 & 0 & 0 
\end{bmatrix}
=
\begin{bmatrix} 
0 & 0 & 1 \\
0 & \frac{1}{5} & 3 \\
1 & 0 & 0 
\end{bmatrix}
$$
$$
\begin{bmatrix} 
0 & 0 & 1 \\
0 & \frac{1}{5} & 3 \\
1 & 0 & 0 
\end{bmatrix}
\begin{bmatrix} 
0 & 0 & 1 \\
-15 & 5 & 0 \\
1 & 0 & 0 
\end{bmatrix}
=
\begin{bmatrix} 
1 & 0 & 0 \\
0 & 1 & 0 \\
0 & 0 & 1 
\end{bmatrix}
$$
This matrix is the inverse of $M$.
}

\item Are there a set of transformations we can apply to $M = 
\begin{bmatrix} 
1 & -2 & -1 \\
0 & 2 & 2 \\
1 & 0 & 1 
\end{bmatrix}$ to make it the identity? If so, what are they? If not, why is is not possible?

\note{
This part is meant to help students understand how a matrix with linearly dependent rows is not invertible.
}


\sol{
No, there are not a set of transformations. It is not possible because the rows are linearly dependent, so you end up with a row of $0$s. 
}

\item Can you find the inverse of a non-square matrix (e.g. a $2 \times 3$ matrix)?

\note{
This question forces the students to think about the requirements and properties of an inverse.
}

\sol{
No, not really. There might be a left or right inverse, but there will not be an inverse. Usually $AA^{-1} = A^{-1}A = I$, but this is not possible with rectangular matrices because their row count and column count differ,
}

\end{enumerate}