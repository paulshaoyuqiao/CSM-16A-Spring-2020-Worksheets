
\begin{enumerate}
\item{What is an inner product?}\\
\sol{An inner product describes a way to multiply vectors, such that the result is a scalar. It is often used to describe properties such as the length of a vector, the angle between vectors, orthogonality of vectors, etc.\\
An inner product must satisfy the following properties:\\
1. Symmetry: $\innp{\vec{x}}{\vec{y}} = \innp{\vec{y}}{\vec{x}}$ \\
2. Homogeneity: $\innp{c\vec{x}}{\vec{y}} = c\innp{\vec{x}}{\vec{y}}$\\
3. Additivity: $\innp{\vec{x}+\vec{y}}{\vec{z}} = \innp{\vec{x}}{\vec{z}} + \innp{\vec{y}}{\vec{z}}$\\
4. Positive-definiteness: $\innp{\vec{x}}{\vec{x}} \geq 0$, and is $= 0$ iff $\vec{x} = \vec{0}$}

\item{What is the dot product between two vectors $\vec{x}, \vec{y} \in \mathbb{R}^n$?}\\
\sol{The dot product is defined as the sum of element-wise products, i.e. \[x_1y_1 + x_2y_2 + \hdots + x_ny_n\]}

In the next four parts, we prove that the dot product is an inner product. Do note that the dot product is simply a type of inner product, and other inner products are also possible.
\item{Prove that the dot product satisfies symmetry, i.e. that $\innp{\vec{x}}{\vec{y}} = \innp{\vec{y}}{\vec{x}}$}\\
\sol{First, we write the definition of a dot product again:
\[x_1y_1 + x_2y_2 + \hdots + x_ny_n\]
Since $x_i$ and $y_i$ are just scalars, and we know that scalar multiplication commutes, we can rewrite this as:
\[y_1x_1 + y_2x_2 + \hdots + y_nx_n = \innp{\vec{y}}{\vec{x}}\]
}

\item{Prove that the dot product satisfies homogeneity, i.e. that $\innp{c\vec{x}}{\vec{y}} = c\innp{\vec{x}}{\vec{y}}$: $c \in \mathbb{R}$}\\
\sol{Writing out $\innp{c\vec{x}}{\vec{y}}$, we have:
\[cx_1y_1 + cx_2y_2 + \hdots + cx_ny_n\]
Since there is a c in every term, we can pull it out, getting:
\[c(x_1y_1 + x_2y_2 + \hdots + x_ny_n) = c\innp{\vec{x}}{\vec{y}}\]
}

\item{Prove that the dot product satisfies additivity, i.e. that $\innp{\vec{x}+\vec{y}}{\vec{z}} = \innp{\vec{x}}{\vec{z}} + \innp{\vec{y}}{\vec{z}}$}

\sol{Writing out $\innp{\vec{x}+\vec{y}}{\vec{z}}$, we get:
\[(x_1+y_1)z_1 + (x_2+y_2)z_2 + \hdots + (x_n+y_n)z_n\]
distributing we get
\[x_1z_1 + x_2z_2 + \hdots + x_nz_n + y_1z_1 + y_2z_2 + \hdots + y_nz_n = \innp{\vec{x}}{\vec{z}} + \innp{\vec{y}}{\vec{z}}\]
}

\item{Prove that the dot product satisfies positive-definiteness, i.e., that $\innp{\vec{x}}{\vec{x}} \geq 0$, and is equal to 0 iff $\vec{x} = \vec{0}$}

\sol{
$\innp{\vec{x}}{\vec{x}} = x_1^2 + x_2^2 + \ldots + x_n^2$. Since each term in this sum is $\geq 0$, $\innp{\vec{x}}{\vec{x}} \geq 0$. Also, $\innp{\vec{x}}{\vec{x}}$ is clearly 0 only when $x_1, x_2, \ldots, x_n = 0$, i.e., $\vec{x} = \vec{0}$
}

We will now consider ways to use dot products to do neat things. For each of the following, assume that you're given a $\vec{x}$, and that you get a pick $\vec{y}$ of your choosing. Describe a $\vec{y}$, such that when you compute $\innp{\vec{x}}{\vec{y}}$, you get:\\
\item{The sum of every element in $\vec{x}$}\\
\sol{We can do this by setting $\vec{y} = \vec{1}$ taking the following dot product:
\[\innp{\vec{1}}{\vec{x}} = 1x_1 + 1x_2 + \hdots + 1x_n\]
}

\item{The sum of certain elements in $\vec{x}$}\\
\sol{We can do this by letting $\vec{y}$ be a vector of 1s and 0s, where the ones are in the positions corresponding to the desired elements.}

\item{The mean of all the items in $\vec{x}$ (for $\vec{x}$ in $\mathbb{R}^n$)}\\
\sol{For this case, we can have some vector $\vec{y}$, where every element is $\frac{1}{n}$, so we have:
\[\innp{\vec{x}}{\vec{y}} = \frac{1}{n}x_1 + \frac{1}{n}x_2 + \hdots + \frac{1}{n}x_n = \frac{1}{n}(x_1 + x_2 + \hdots + x_n)\]
}

\item{The sum of the elements of $\vec{x}$ squared}\\
\sol{For this case, we can just take the dot product of x with itself, 
\[\innp{\vec{x}}{\vec{x}} = x_1^2 + x_2^2 + \hdots + x_n^2\]
}

We will conclude by making some observations based on that last case.\\

\item{Consider that last case, where we summed the squares of the elements of a vector. Try doing that for a few 2-dimensional vectors (vectors of length 2). What do you notice about the resulting answer? What about for vectors of length 3, or for vectors of any length n?}\\
\sol{After trying out a few examples, you may notice that the dot product of a vector with itself is the square of the length of the vector! Another way to see this is to think of the normal euclidean distance equation:
\[d = \sqrt{(x_1 - y_1)^2 + (x_2 - y_2)^2}\]
This can be generalized to any number of dimensions. Now, consider that $\vec{y}$ is a vector of all zeroes, we now have an equation which is exactly the square root of $\innp{\vec{x}}{\vec{x}}$, which we also name the $\ell$2-norm of $\vec{x}$, or $\norm{x}_2$, or also $\norm{x}$.
}

\end{enumerate}