
Suppose we have a symmetric matrix $A \in \mathbb{R}^{n \times n}$.

\begin{enumerate}
    \item Show that A has all real eigenvalues.

    \sol{

        Without loss of generality, let $(\lambda, \vec{v})$ be any eigenvalue-vector pair of $A$.

        We have $A\vec{v} = \lambda \vec{v}$.

        Since $A$ is also symmetric, $A = A^T$.

        Consider the expression $\vec{v}^T A^T A\vec{v}$, we have:
        $$\vec{v}^T A^T A \vec{v} = (A\vec{v})^T A\vec{v} = \innp{A\vec{v}}{A\vec{v}} = \norm{A\vec{v}}^2$$

        At the same time, using what we have shown in part 1 of problem 3,
        $$\vec{v}^T A^T A \vec{v} = \vec{v}^T A^2 \vec{v} = \vec{v}^T \lambda^2 \vec{v} = \lambda^2 \vec{v}^T \vec{v} = \lambda^2 \norm{\vec{v}}^2$$

        We can see:
        $$\norm{A\vec{v}}^2 = \lambda^2 \norm{\vec{v}}^2$$
        $$\lambda^2 = \frac{\norm{A\vec{v}}^2}{\norm{\vec{v}}^2}$$
        Since $\norm{A\vec{v}}^2 > 0, \norm{\vec{v}}^2 > 0$, we can see that $\lambda^2 = \text{ some positive number}$.

        Hence, $\lambda$ must be real.
    }

    \item{
        Using the result from part 2 of probelm 3, explain why the eigenvectors of $A$ are orthogonal to each other. (If the set of all eigenvectors are orthogonal to each other, we call the set an \textit{orthogonal eigenbasis})
    }

    \sol{
        Since $A$ is symmetric, $A = A^T$. Suppose $\vec{u}$ is an eigenvector of $A$ with associated eigenvalue $\alpha$, and $\vec{v}$ is another eigenvector of $A$ with associated eigenvalue $\beta$.

        Slightly modifying the proof from part 2 of problem 3, we can see that 
        $$A\vec{u} = \alpha \vec{u},$$
        $$A\vec{v} = A^T \vec{v} = \beta\vec{v}.$$ 
        Now the rest of the proof from part 2 of problem 3 follows.
    }

    
\end{enumerate}