% Author: Pranshu Bansal
% Email: pranshu@berkeley.edu
\newcommand{\Amat}{\ensuremath{\begin{bmatrix}
4 & 1  \\
3 & 2
\end{bmatrix}}}

\note{Description: Meant to be a review of eigenvectors, eigenvalues and transformations.}

\begin{enumerate}
\item {Suppose $\lambda$ is an eigenvalue for the matrix A. Consider the $\lambda$-eigenspace of A: the set of all vectors v satisfying the equation $\mathbf{A}\vec{v} = \lambda\vec{v}$. Show that this eigenspace is a subspace by directly checking the three conditions needed to be a subspace.}

\note{
    This gives students some review of eigenvalues, eigenvectors and subspaces. It might help to start this question by prompting students to state what the three conditions are to prove that the eigenspace is a subspace.
}

\answerbox{5cm}

\sol{
First, we have to check that $\vec{0}$ is in the subspace: this is true because $\mathbf{A}\vec{0} = \lambda\vec{0} = \vec{0}$ (regardless of what the eigenvalue $\lambda$ is). \\

Next, suppose $\vec{u}$ and $\vec{v}$ are in the subspace. This means that: \\

$$\mathbf{A}\vec{u} = \lambda\vec{u}$$
$$\mathbf{A}\vec{v} = \lambda\vec{v}$$ \\
$$\mathbf{A}(\vec{u} + \vec{v}) = \mathbf{A}\vec{u} + \mathbf{A}\vec{v} = \lambda\vec{u} + \lambda\vec{v} = \lambda(\vec{u} + \vec{v})$$ \\

This means $\vec{u} + \vec{v}$ is also in the subspace.

Finally, suppose $\vec{v}$ is in the subspace and $r$ is a scalar. Then, \\ 
$$\mathbf{A}(r\vec{v}) = r(\mathbf{A}\vec{v}) = r(\lambda\vec{v}) = \lambda(r\vec{v})$$

This means that $r\vec{v}$ is also in the subspace. \\

Since the eigenspace satisfies all three conditions of being a subspace, we can say that it is a subspace.
}

\item{
    Solve for the eigenvalue-eigenvector pairs for the following 2 by 2 matrix: \\
    $$
    \mathbf{A} = \Amat
    $$
}

\note{
    This is a mechanical question, if you feel that your students are comfortable with solving for eigenvalues and eigenvectors you can consider skipping this part.
}

\answerbox{7cm}

\sol{
    To solve for eigenvalues and eigenvectors, let's go back and review the definition of eigenvectors and eigenvalues:
    \\
    If $\vec{x}$ and $\lambda$ are the eigenvector and eigenvalue of $\mathbf{A}$, respectively, then the following equation holds:
    
    $$\mathbf{A}\vec{x} = \lambda\vec{x}$$
    
    Since the (appropriately sized) identity matrix is analogous to multiplying by 1 in arithmetic, we can say:

    $$\mathbf{A}\vec{x} = \lambda \mathbf{I} \vec{x}$$
    
    Rearranging, we get:
    
    $$\mathbf{A}\vec{x} - (\lambda \mathbf{I}) \vec{x} = \vec{0}
    $$
    $$
    (\mathbf{A} - \lambda \mathbf{I})\vec{x} = \vec{0}
    $$
    
    What does this look like? It looks similar to solving for the nullspace of $(\mathbf{A} - \lambda \mathbf{I})$!
    
    Assuming that there is a nontrivial nullspace, that also means that $\mathbf{det}(\mathbf{A} - \lambda \mathbf{I}) = 0$!
    
    Let's solve for $\lambda$ first:
    
    $$(\mathbf{A} - \lambda \mathbf{I}) = \Amat - \begin{bmatrix}
    \lambda & 0 \\
    0 & \lambda
    \end{bmatrix}
    $$
    $$= \begin{bmatrix}
    4 - \lambda & 1 \\
    3 & 2 - \lambda
    \end{bmatrix}$$
    $$\mathbf{det}(\mathbf{A} - \lambda \mathbf{I}) = (4 - \lambda)(2 - \lambda) - 3$$
    $$= 5 - 6\lambda + \lambda^2$$
    $$= (\lambda - 5)(\lambda - 1)$$
    By factoring:
    $$\lambda = 5, 1$$
    
    Let's check: We've just solved for the eigenvalues. But what about the eigenvectors? 
    
    To do that, we plug in $\lambda$ into $(\mathbf{A} - \lambda \mathbf{I})$ and solve for the nullspace!
    
    For $\lambda = 5$:
    
    $$
    (\mathbf{A} - \lambda \mathbf{I})\vec{x} = \vec{0}
    $$
    $$\begin{bmatrix}
    -1 & 1 \\
    3 & -3
    \end{bmatrix}\vec{x} = \vec{0}$$
    
    We can see that eigenvector $\begin{bmatrix} 1 \\ 1 \end{bmatrix}$ spans the nullspace of the above matrix.

    So the first pair is $$\lambda = 5, \begin{bmatrix} 
    1 \\
    1
    \end{bmatrix}$$ \\

    Repeating for $\lambda = 1$,
    $$
    (\mathbf{A} - \lambda \mathbf{I})\vec{x} = \vec{0}
    $$ 
    $$\begin{bmatrix}
    3 & 1 \\
    3 & 1
    \end{bmatrix}\vec{x} = \vec{0}$$
    
    We can see that eigenvector $\begin{bmatrix} -1 \\ 3 \end{bmatrix}$ spans the nullspace of the above matrix.
    
    So, the second pair is
    $$\lambda = 1, 
    \begin{bmatrix} 
    -1 \\
    3
    \end{bmatrix}$$
    }

\item{
    Projection of a vector $\vec{u}$ onto $\vec{v}$ is given by:

    $$\frac{\vec{u} \cdot \vec{v}}{{||\vec{v}||}^2} \vec{v}$$

    Prove that projection onto a vector $\vec{v}$ is a linear transformation.
}

\note{
    Students might ask where this formula is derived from, it helps to be prepared for this.
}

\answerbox{7cm}

\sol{
    Let us represent this transformation using P.

    $$P(\vec{a}) = \frac{\vec{a} \cdot \vec{v}}{{||\vec{v}||}^2} \vec{v}$$

    Let's check if it satisfies the condition of linearity.

    $$P(\vec{a} + \vec{b}) = \frac{(\vec{a} + \vec{b}) \cdot \vec{v}}{{||\vec{v}||}^2} \vec{v}$$
    $$P(\vec{a} + \vec{b}) = \frac{\vec{a} \cdot \vec{v}}{{||\vec{v}||}^2} \vec{v} + \frac{\vec{b} \cdot \vec{v}}{{||\vec{v}||}^2} \vec{v}$$
    $$P(\vec{a} + \vec{b}) = P(\vec{a}) + P(\vec{b})$$

    Hence, the projection transformation satisfies additivity. Let's check if it satisfies the condition of scalar multiplication.

    $$P(r\vec{a}) = \frac{(r\vec{a}) \cdot \vec{v}}{{||\vec{v}||}^2} \vec{v}$$
    $$P(r\vec{a}) = r \cdot \frac{(\vec{a}) \cdot \vec{v}}{{||\vec{v}||}^2} \vec{v}$$
    $$P(r\vec{a}) = r \cdot P(\vec{a})$$

    Hence, the projection transformation is a linear transformation as it satisfies both the conditions - vector addition and scalar multiplication.
}

\end{enumerate}