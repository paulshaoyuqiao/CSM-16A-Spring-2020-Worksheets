Suppose we have a matrix $A \in \mathbb{R}^{n \times n}$.

\begin{enumerate}
    \item Show that if $\vec{v}$ is an eigenvector of $A$, then it must also be an eigenvector of $A^{2}$.

    \sol{
        Suppose $A\vec{v} = \lambda \vec{v}$. Left multiply both sides by $A$, we have $A^2\vec{v} = A\cdot \lambda \vec{v}$. Since $\lambda$ is a constant, we can switch its position with $A$ on the right side. This gives us:
        $$A^2\vec{v} = \lambda \cdot A\vec{v} = \lambda\cdot\lambda\vec{v} = \lambda^2 \vec{v}.$$
    }

    \item Show that if $\vec{u}$ is an eigenvector of $A$ with associated eigenvalue $\alpha$, and $\vec{v}$ is an eigenvector of $A^T$ with associated eigenvalue $\beta$, if $\alpha \neq \beta$, then $\vec{u}$ and $\vec{v}$ must be orthogonal to each other.

    \sol{
        From what's given in the question, we know that:
        $$A\vec{u} = \alpha \vec{u},$$
        $$A^T\vec{v} = \beta \vec{v}.$$
        To show $\vec{u}$ and $\vec{v}$ are orthogonal to each other, we must show that $\vec{u}^T \vec{v} = 0$.

        Since we have:
        $$A\vec{u} = \alpha \vec{u},$$
        Left multiply the first equation by $\vec{v}^T$. This gives us:
        $$\vec{v}^T A\vec{u} = \vec{v}^T \alpha \vec{u} = \alpha \vec{v}^T \vec{u}$$
        At the same time, note the following:
        $$\vec{v}^T A\vec{u} = (A^T \vec{v})^T \vec{u} = (\beta \vec{v})^T \vec{u} = \beta \vec{v}^T \vec{u}$$
        Therefore, we can see that:
        $$\alpha \vec{v}^T \vec{u} = \beta \vec{v}^T \vec{u}$$
        $$ (\alpha - \beta) \vec{v}^T \vec{u} = 0$$
        Since $\alpha \neq \beta$, $\alpha - \beta \neq = 0$, then it must be that $\vec{v}^T \vec{u} = \vec{u}^T \vec{v} = 0$.

        Therefore, $\vec{u}$ and $\vec{v}$ must be orthogonal to each other.
    }

    \vspace{0.5cm}
    \textbf{For the following parts, assume $A$ is also symmetric.}

    \item Show that A has all real eigenvalues.
    
        \sol{
    
            Without loss of generality, let $(\lambda, \vec{v})$ be any eigenvalue-vector pair of $A$.
    
            We have $A\vec{v} = \lambda \vec{v}$.
    
            Since $A$ is also symmetric, $A = A^T$.
    
            Consider the expression $\vec{v}^T A^T A\vec{v}$, we have:
            $$\vec{v}^T A^T A \vec{v} = (A\vec{v})^T A\vec{v} = \innp{A\vec{v}}{A\vec{v}} = \norm{A\vec{v}}^2$$
    
            At the same time, using what we have shown in part 1 of problem 3,
            $$\vec{v}^T A^T A \vec{v} = \vec{v}^T A^2 \vec{v} = \vec{v}^T \lambda^2 \vec{v} = \lambda^2 \vec{v}^T \vec{v} = \lambda^2 \norm{\vec{v}}^2$$
    
            We can see:
            $$\norm{A\vec{v}}^2 = \lambda^2 \norm{\vec{v}}^2$$
            $$\lambda^2 = \frac{\norm{A\vec{v}}^2}{\norm{\vec{v}}^2}$$
            Since $\norm{A\vec{v}}^2 > 0, \norm{\vec{v}}^2 > 0$, we can see that $\lambda^2 = \text{ some positive number}$.
    
            Hence, $\lambda$ must be real.
        }
    
        \item{
            Using the result from part 2, explain why the eigenvectors of $A$ are orthogonal to each other. (If the set of all eigenvectors are orthogonal to each other, we call the set an \textit{orthogonal eigenbasis})
        }
    
        \sol{
            Since $A$ is symmetric, $A = A^T$. Suppose $\vec{u}$ is an eigenvector of $A$ with associated eigenvalue $\alpha$, and $\vec{v}$ is another eigenvector of $A$ with associated eigenvalue $\beta$.
    
            Slightly modifying the proof from part 2 of problem 3, we can see that 
            $$A\vec{u} = \alpha \vec{u},$$
            $$A\vec{v} = A^T \vec{v} = \beta\vec{v}.$$ 
            Now the rest of the proof from part 2 follows.
        }
    
\end{enumerate}