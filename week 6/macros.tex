\documentclass[letter]{article}
% Set target color model to RGB
\usepackage[inner=2.0cm,outer=2.0cm,top=2.5cm,bottom=2.5cm]{geometry}
\usepackage{setspace}
\usepackage[dvipsnames]{xcolor}
\usepackage{verbatim}
\usepackage{subcaption}
\usepackage{biblatex}
\usepackage{tkz-euclide}
\usepackage{pdfpages}
% \usepackage{physics}
\usepackage{pgfplots}
\usepackage[american,siunitx]{circuitikz}
\usepackage{amsgen,amsmath,amstext,amsbsy,amsopn,tikz,amssymb,tkz-linknodes}
\usepackage{float}
\usepackage{fancyhdr}
\usepackage[colorlinks=true, urlcolor=blue,  linkcolor=blue, citecolor=blue]{hyperref}
\usepackage[colorinlistoftodos]{todonotes}
\usepackage{rotating}
\usepackage{booktabs}

\hypersetup{%
pdfauthor={Paul Shao},%
pdftitle={Worksheet},%
pdfkeywords={Tikz,latex,bootstrap,uncertaintes},%
pdfcreator={PDFLaTeX},%
pdfproducer={PDFLaTeX},%
}

\newcommand{\ra}[1]{\renewcommand{\arraystretch}{#1}}

\newtheorem{thm}{Theorem}[section]
\newtheorem{prop}[thm]{Proposition}
\newtheorem{lem}[thm]{Lemma}
\newtheorem{cor}[thm]{Corollary}
\newtheorem{defn}[thm]{Definition}
\newtheorem{rem}[thm]{Remark}
\numberwithin{equation}{section}

\newcommand{\homework}[5]{
   \pagestyle{myheadings}
   \thispagestyle{plain}
   \newpage
   \setcounter{page}{1}
   \noindent
   \begin{center}
   \framebox{
      \vbox{\vspace{2mm}
    \hbox to 6.28in { {\bf CSM 16A \hfill {\small #2}} }
       \vspace{6mm}
       \hbox to 6.28in { {\LARGE \hfill #1  \hfill} }
       \vspace{6mm}
       \hbox to 6.28in { {\it Term: {\rm #3} \hfill Name: \hspace{2cm}{\rm #4}}}
      \vspace{2mm}}
   }
   \end{center}
   \markboth{#5 -- #1}{#5 -- #1}
   \vspace*{4mm}
}

\newcommand{\problem}[1]{~\\\fbox{\textbf{Problem #1}}\hfill}
\newcommand{\subproblem}[1]{~\newline\textbf{(#1)}}
\newcommand{\D}{\mathcal{D}}
\newcommand{\Hy}{\mathcal{H}}
\newcommand{\VS}{\textrm{VS}}
\newcommand{\solution}{~\newline\textbf{\textit{(Solution)}} }

\newcommand{\bbF}{\mathbb{F}}
\newcommand{\bbX}{\mathbb{X}}
\newcommand{\bI}{\mathbf{I}}
\newcommand{\bX}{\mathbf{X}}
\newcommand{\bY}{\mathbf{Y}}
\newcommand{\bepsilon}{\boldsymbol{\epsilon}}
\newcommand{\balpha}{\boldsymbol{\alpha}}
\newcommand{\bbeta}{\boldsymbol{\beta}}
\newcommand{\0}{\mathbf{0}}

\newcommand{\sol}[1]{{\color{blue} \textbf{Solution: } #1}} % solutions in blue
\newcommand{\learning}[1]{{\color{ForestGreen} \textbf{Learning Goal: } #1}}
\newcommand{\note}[1]{{\color{purple} \textbf{Meta: } #1}} % notes in green
\newcommand{\answerbox}[1]{
   \noindent\fbox{
      \parbox{16cm}{
          \hspace{16cm}
          \vspace{#1}
      }
  }
}
\newcommand{\statement}[1]{
   \noindent\fbox{
      \parbox{16cm}{
          #1
      }
  }
}
\def\m{\ensuremath\mathbf}