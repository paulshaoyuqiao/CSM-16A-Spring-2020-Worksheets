% Lydia Lee, Spring 2019
% lydia.lee@berkeley.edu
% Paul Shao, Spring 2019
% paulshaoyuqiao1@berkeley.edu

\qns{Subspace Drills}

Determine if the following sets of vectors describe subspaces.

\sol{
\begin{itemize}
    \item Emphasize the importance of \textit{generalization} when determining whether a given set is a subspace (and for proofs in general). It is crucial to show that the two properties above hold for \textit{all} vectors within the set, not just one or two numerical examples.
    \item Clarify the mathematical notation, e.g. $\forall$ means ``for all'' and $:$ means ``such that''. Vocalizing what you're writing down as you write it will help keep you from moving too quickly while clarifying any confusing notation. 
    \item Graphical intuition tends to help students who don't have a strong linear algebra background.
\end{itemize}}

\begin{enumerate}
\qitem\label{ques:trivial}{
    $\{\vec{0}\}$
}

\sol{
    When teaching this question to your students, make sure you mention the significance of a single-element set (the fact that $\vec{0}$ is the \textbf{only} vector in the set), which greatly simplifies the subspace test process.}

\ans{
    \begin{itemize}
        \item \textit{Vector addition}: Since $\vec{0}$ is the only vector in this set, $\vec{0} + \vec{0} = \vec{0} \in V$.
        \item \textit{Scalar multiplication}: Since $a \cdot \vec{0} = \vec{0} \: \: \forall a \in \mathbb{R}$, we have that $a \cdot \vec{0} \in V$.
    \end{itemize}

    The set is closed under vector addition and scalar multiplication, and so it is a vector space! $\square$
}

\qitem\label{ques:twoVecs}{
    $\left\{\begin{bmatrix}1\\0\end{bmatrix}, \begin{bmatrix}0\\1\end{bmatrix}\right\}$
}

\sol{
    Emphasize the difference between a set containing 2 vectors and a set that is the \textbf{span} of 2 vectors. The former \textbf{only contains 2 vectors}, while the latter \textbf{contains all linear combinations of the 2 vectors}.}

\ans{
    \begin{itemize}
        \item \textit{Vector addition}: We can see that 
            $$\begin{bmatrix}1\\0\end{bmatrix} + \begin{bmatrix}0\\1\end{bmatrix} = \begin{bmatrix}1\\1\end{bmatrix}$$
            which is not in the set of vectors.
        \item \textit{Scalar multiplication}: Taking either of the vectors and scaling them by $\alpha \neq 1$ returns a vector not in the set, and so it is not closed under scalar multiplication.
    \end{itemize}

    This is \textit{not} a vector subspace. $\square$
}

\qitem\label{ques:spanTwoVecs}{
    span$\left(\begin{bmatrix}1\\0\end{bmatrix}, \begin{bmatrix}0\\1\end{bmatrix}\right)$
}

\sol{
    Reiterate the distinction between a single vector (like what's seen in part \ref{ques:twoVecs}) and its span.}

\ans{
    By definition, the span of $N$ vectors is the set of the linear combinations of those $N$ vectors. For this problem, we can rewrite the set as:
    $$\left\{a\begin{bmatrix}1\\0\end{bmatrix} + b\begin{bmatrix}0\\1\end{bmatrix} = \begin{bmatrix}a\\b\end{bmatrix}: a, b \in \mathbb{R} \right\}$$

    With the new definition, we see that the span of this set is $\mathbb{R}^2$, which is a vector subspace. For the sake of completeness, however, we'll go through the standard tests:
    \begin{itemize}
        \item \textit{Vector addition}: Let's define the following $\vec{v_1}$ and $\vec{v_2}$, both in the set.
        \begin{align*}
            \vec{v_i} &= a_i\begin{bmatrix}1\\0\end{bmatrix} + b_i\begin{bmatrix}0\\1\end{bmatrix}, i=1, 2\\
            \vec{v_1} + \vec{v_2} &= \left(a_1\begin{bmatrix}1\\0\end{bmatrix} + b_1\begin{bmatrix}0\\1\end{bmatrix}\right) + \left(a_2\begin{bmatrix}1\\0\end{bmatrix} + b_2\begin{bmatrix}0\\1\end{bmatrix}\right)\\
                &= \left(a_1 + a_2\right)\begin{bmatrix}1\\0\end{bmatrix} + \left(b_1 + b_2\right)\begin{bmatrix}0\\1\end{bmatrix}\\
                &= a\begin{bmatrix}1\\0\end{bmatrix} + b\begin{bmatrix}0\\1\end{bmatrix}
        \end{align*}
        \item \textit{Scalar multiplication}:
        \begin{align*}
            \vec{v} &= a\begin{bmatrix}1\\0\end{bmatrix} + b\begin{bmatrix}0\\1\end{bmatrix}\\
            \alpha\vec{v} &= \alpha a\begin{bmatrix}1\\0\end{bmatrix} + \alpha b\begin{bmatrix}0\\1\end{bmatrix}\\
                &= \hat{a}\begin{bmatrix}1\\0\end{bmatrix} + \hat{b}\begin{bmatrix}0\\1\end{bmatrix}
        \end{align*}
    \end{itemize}

    Hence, the set is a vector subspace $\square$
}

\qitem\label{ques:affine}{
    $\left\{\begin{bmatrix}x_1\\x_2\\x_3\end{bmatrix} \in \mathbb{R}^3 : x_1 + x_2 + x_3 = 1\right\}$
}

\sol{
    Ask your students about the graphical interpretation of this set, then draw it out.
    
    Make sure to mention to the students that since geometrically the plane $x_1 + x_2 + x_3 = 1$ doesn't pass through the origin, it doesn't contain the zero vector, and thereby is not a subspace.
}

\ans{


    \textbf{Graphical Understanding}

    This set describes all points on the plane that passes through $(1,0,0)$, $(0,1,0)$, and $(0,0,1)$.
    \begin{center}
        \begin{tikzpicture}[scale=1,tdplot_main_coords]
            \filldraw[
                draw=red,%
                fill=red!20,%
            ]          (1,0,0)
                    -- (0,1,0)
                    -- (0,0,1)
                    -- cycle;
            \draw[thick,->] (0,0,0) -- (2,0,0) node[anchor=north east]{$x_1$};
            \draw[thick,->] (0,0,0) -- (0,2,0) node[anchor=north west]{$x_2$};
            \draw[thick,->] (0,0,0) -- (0,0,2) node[anchor=south]{$x_3$};
        \end{tikzpicture}
    \end{center}

    (If it's not immediately clear why this is true, try thinking of the line $x_1 + x_2 = 1$.)

    We know from the properties of vector spaces that one implication of closure under addition and scalar multiplication is that the vector space must contain the zero vector. Furthermore, using any two vectors in the space and adding them together disproves closure under vector addition; scaling a point in the plane by $\alpha \neq 1$ disproves closure under scalar multiplication.

    \textbf{Quick Counterexample}
    Using the endpoints $$\begin{bmatrix}1\\0\\0\end{bmatrix}, \begin{bmatrix}0\\1\\0\end{bmatrix}$$ and testing closure under scalar multiplication and addition is sufficient.
}

\qitem\label{ques:notAffine}{
    $\left\{\begin{bmatrix}x_1\\x_2\\x_3\end{bmatrix} \in \mathbb{R}^3 : x_1 + x_2 + x_3 = 0\right\}$
}

\sol{

    The process for finding the spanning vectors is the same as what they'll have seen in discussion for finding the null space.
    
}

\ans{
    Similar to part \ref{ques:affine}, we can gain some understanding of what this is by considering the fact that it describes a plane which passes through the origin. However, this is not sufficient to prove that this (spoiler alert) is a vector space. We can do this by seeing if we can find a set of vectors whose span is the set.

    The process is identical to finding a null space given a set of linear equations.

    Looking at the equation we're given
    $$x_1 + x_2 + x_3 = 0$$
    we have one equation but three unknowns, meaning we can choose two free variables. Any two will work, but we'll say that $x_1$ and $x_2$ are free. That means we want to write $x_3$ in terms of the other two variables:
    $$x_3 = -x_1 - x_2$$
    So rewriting the original vector in terms of our free variables $x_1$ and $x_2$:
    \begin{align*}
        \begin{bmatrix}x_1\\x_2\\x_3\end{bmatrix} &= \begin{bmatrix}x_1\\x_2\\-x_1-x_2\end{bmatrix}\\
            &= x_1\begin{bmatrix}1\\0\\-1\end{bmatrix} + x_2\begin{bmatrix}0\\1\\-1\end{bmatrix}
    \end{align*}
    We can now rewrite the set as span$\left(\begin{bmatrix}1&0&-1\end{bmatrix}^T, \begin{bmatrix}0&1&-1\end{bmatrix}^T\right)$. We know from part \ref{ques:spanTwoVecs} that this set is indeed a vector space! $\square$
}
\end{enumerate}
