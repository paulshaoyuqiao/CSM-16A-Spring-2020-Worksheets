% Author: Paul Shao
% Email: paulshaoyuqiao1@berkeley.edu

For each of the following statements, determine if they are \textbf{TRUE} or \textbf{FALSE}. If they are \textbf{FALSE}, try to come up with a counterexample; if they are \textbf{TRUE}, give a brief explanation.

\note{When walking through every statement in this question, be sure to encourage the students to think from the groud up (beginning with definitions). It is very tempting to try to fit brute-force counterexample; but in some cases, it might be very hard to directly come up with them without understanding the concept first.}
\begin{enumerate}
\item If the augmented matrix of the linear system represented by $A\vec{x} = \vec{b}$ has a pivot in the last column, then the matrix vector equation $A\vec{x} = \vec{b}$ has no solution.
    
    \sol{TRUE. A pivot in the last column of an augmented matrix is equivalent to having a row of the following form $\begin{bmatrix}
0 & 0 & 0 & \dots & | & b
\end{bmatrix}$, where $b \neq 0$. In this case, there is no possible solution that would make the equivalent equation for this row hold.}

    \item If A is a 3 $\times$ 3 matrix such that the matrix vector equation $A\vec{x} = \vec{0}$ has only the trivial solution ($\vec{x} = \vec{0}$), then the matrix vector equation $A\vec{x} = \vec{b}$ is consistent for every vector $\vec{b}$ in $\mathbb{R}^3$
    
    \sol{TRUE. If $A\vec{x} = \vec{0}$ has only the trivial solution $\vec{x} = \vec{0}$, then it implies that the dimension of the null space in $A$ must be 0 (or equivalently, $A$ can be reduced to a row echelon form such that there is a pivot in every column). In this case, the column vectors in $A$ must span all of $\mathbb{R}^3$, and consequently, the matrix  vector equation $A\vec{x} = \vec{b}$ is consistent for every vector $\vec{b}$ in $\mathbb{R}^3$.}
    
    \item If the matrix vector equation $A\vec{x} = \vec{0}$ is true only when $\vec{x} = \vec{0}$, then the matrix $A$ has an inverse ($A$ is invertible).
    
    \sol{FALSE. $A$ must also be a square matrix. Consider the matrix $$A = \begin{bmatrix}
1 & 0\\ 
0 & 1\\ 
0 & 0
\end{bmatrix}$$ This matrix has only $\vec{0}$ as the solution to $$A\vec{x} = \vec{0}$$; however, since it is not square, we cannot find an inverse for it.}

    \item A matrix $A$ is called $\textit{symmetric}$ if it is equal to its transpose: $A = A^T$. If $A$ is an invertible and \textit{symmetric} matrix, $A^{-1}$ must also be \textit{symmetric}.
    
    \sol{TRUE. We want to show that $A^{-1} = (A^{-1})^T$. 
    $(A^{-1})^T = (A^T)^{-1} = A^{-1}$. Hence, $A^{-1}$ is also \textit{symmetric}. 
    
    \textbf{Note}: To show why $(A^{-1})^T = (A^T)^{-1}$, consider:
    $$A^T(A^{-1})^T = (A^{-1}A)^T = I^T = I$$
    $$A^T(A^T)^{-1} = I$$
    Hence, $$A^T(A^{-1})^T = A^T(A^T)^{-1}, A^T((A^{-1})^T - A^T(A^T)^{-1}) = 0$$
    Since $A^T$ can be any matrix, it must true that $(A^{-1})^T - (A^T)^{-1} = 0 \longrightarrow (A^{-1})^T = (A^T)^{-1}$
    }

\end{enumerate}