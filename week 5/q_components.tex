\note{
Introduction to basic circuit components.
Mentors: do a mini lecture on what is charge, what is voltage, and what is current. See lecture notes for definitions. 
}

In this problem, we will introduce the fundamental circuit components. 

\begin{enumerate}

\item{
What is a voltage source?
}

\sol{
Firstly, a voltage source is represented in this manner: 
\begin{center}
    \begin{circuitikz}
        \draw(0,0)
	    to[V=$V$] ++(1,0);
    \end{circuitikz}
\end{center}

A voltage source \textbf{guarantees} that the potential at its positive end will be $V$ more than the potential at its negative end, no matter what. 
}

\item{
 What is a current source?  
}

\sol{
A current source is represented in this manner: 
\begin{center}
    \begin{circuitikz}
        \draw(0,0)
	     to[I, l=$I$] ++(1,0);
    \end{circuitikz}
\end{center}
A current source \textbf{guarantees} that the current passing through the unit in the direction of the arrow will be its designated value. 
}

\item{What is voltage? What is a voltage drop?}

\sol{
For our discussion, it suffices to think of voltage as a kind of driver for current. Current is the movement of charges. A voltage difference forces current to move from the point (node) that has higher voltage, to the point that has lower voltage. 

Voltage drop is the voltage lost (decline of nodal voltage) across a circuit component. 
}

\item{Consider the figure below. If $V_1 \neq V_2$, what will happen to the circuit?

\begin{center}
    \begin{circuitikz}
    \draw(0,4)
	%to[short] ++(0,-1)
	to[V_=$V_1$] ++(0,-4)
	to[short] node[ground] {} ++(0,-1);
	
	\draw(0,4)
	to[short] ++(2,0)
	to[V_=$V_2$] ++(0,-4)
	to[short] ++(-2,0);
    \end{circuitikz}
\end{center}

}

\sol{
Let us designate the potential at the positive end of $V_1$ to be $V^+_1$, the potential at the negative end of $V_1$ to be $V^-_1$, the potential at the positive end of $V_2$ to be $V^+_2$, and the potential at the negative end of $V_2$ to be $V^-_2$. $V^-_1$ and $V^-_2$ are equal to 0 because of the ground. Then, the potential across $V_1$ is $V^+_1$, and the potential across $V_2$ is $V^+_2$. Since $V^+_1$ and $V^+_2$ are connected by a wire, they must be the same voltage; we know that a wire does not affect a circuit's behavior, so the voltage must stay constant across it. 
This means that $V^+_1 = V^+_2$. However, we know that the voltage potential $V^+_1 - V^-_1$ is not equal to $V^+_2 - V^-_1$ as given in the question. Hence, we see that we cannot have two voltage sources connected in this configuration.
}

\item {What happens in this case if $I_1 \neq I_2$?
\begin{center}
    \begin{circuitikz}
    \draw(0,0) 
    to[I=$I_1$] ++(0,2)
    to[I=$I_2$] ++(0,2);
    \end{circuitikz}
\end{center}
}

\sol{
The current source at the bottom guarantees that through that wire there will be $I_1$ current going through, and the current source at the top guaranteed that $I_2$ current goes through that wire. This is a contradiction, and is not theoretically possible in a circuit. 

Also, look at the point in between the two current sources. $I_1$ enters on one end, and $I_2$ leaves on the other end. This is impossible.
}

\item{
What is a resistor? 
}

\sol{
A resistor is represented in this manner: 
\begin{center}
	\begin{circuitikz}

	\draw(0,0)
	to[R, v=$ $, i=$ $] ++(3,0);
	\end{circuitikz}
	\end{center}
A resistor is a circuit unit designed to `resist' the flow of current. Following convention, there is a ``voltage drop'' across a resistor from the positive end to the negative end. The voltage drop across a resistor is $V_R = I_R R$, where $V_R$ is the voltage drop, $I_R$ is the current through the resistor and $R$ is the resistance of the resistor. 
}

\item{What is power?}

\sol{
Power is the rate at which work is done, where work is in terms of electrical energy.

For circuits, the power \textit{consumed} or \textit{dissipated} by a device is $P = IV$, where the current and voltage abide by passive sign convention.

\textbf{Common Misconceptions:}
\begin{itemize}
	\item Active components do not necessarily dissipate negative power! Consider the following circuit:
	\begin{center}
		\begin{circuitikz}
			\draw
			(0,0) node[ground] () {}
				to[V=$1\si{\volt}$, invert] ++(0,2)
				to[R=$1\si{\ohm}$] ++(2,0)
			(2,0) node[ground] () {}
				to[V=$2\si{\volt}$, invert] ++(0,2);
		\end{circuitikz}
	\end{center}
	When calculating the power dissipated by the LHS voltage source, we see the current flows counterclockwise about the circuit. With passive sign convention, we calculate the power $P_{V_1} = 1\si{\volt} \times 1\si{\ampere}$, which is positive! The left-side voltage source is dissipating power.
\end{itemize}
}

\end{enumerate}