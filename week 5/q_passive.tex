% Lydia Lee, Spring 2019
% lydia.lee@berkeley.edu
For the following components, label all the missing $V_\text{element}$, $I_\text{element}$, and +/- signs. \textit{Hint: The value of the voltage and current sources shouldn't affect passive sign convention---remember that voltage and current can be negative!}

\note{For parts (a) and (b), make sure to clarify to your students that the box figure can represent any arbitrary circuit element (a resistor, a voltage source, etc.).}

\begin{enumerate}
\item{
\begin{center}
	\begin{circuitikz}[scale=0.75]
		\ctikzset{resistor = european}
		\draw (0,0) to[R, v=$V_\text{element}$] ++(4,0);
	\end{circuitikz}
\end{center}
}
\sol{
\begin{center}
	\begin{circuitikz}[scale=0.75]
		\ctikzset{resistor = european}
		\draw (0,0) to[R, v=$V_\text{element}$, i=$I_\text{element}$] ++(5,0);
	\end{circuitikz}
\end{center}
}

\item{
\begin{center}
	\begin{circuitikz}[scale=0.75]
		\ctikzset{resistor = european}
		\draw (0,0) to[R, i=$I_\text{element}$] ++(5,0);
	\end{circuitikz}
\end{center}
}
\sol{
\begin{center}
	\begin{circuitikz}[scale=0.75]
		\ctikzset{resistor = european}
		\draw (0,0) to[R, v=$V_\text{element}$, i=$I_\text{element}$] ++(5,0);
	\end{circuitikz}
\end{center}
}

\item{
\begin{center}
	\begin{circuitikz}[scale=0.75]
		\draw 
		(0,0) to[V, v=$V_\text{S}$, invert] ++(0,3)
			to[open] ++(.5,0)
			to[open, v^=$V_\text{element}$] ++(0,-3);
	\end{circuitikz}
\end{center}
}
\sol{
\begin{center}
	\begin{circuitikz}[scale=0.75]
		\draw 
		(0,0) to[V, v=$V_\text{S}$, i=$I_\text{element}$, invert] ++(0,3)
			to[open] ++(.5,0)
			to[open, v^=$V_\text{element}$] ++(0,-3);
	\end{circuitikz}
\end{center}
}

\item{
\begin{center}
	\begin{circuitikz}[scale=0.75]
		\draw 
		(0,0) to[V, v=$1\si{\volt}$, invert] ++(0,3)
		(0,0) to[open] ++(.5,0)
			to[open, v_=$V_\text{element}$] ++(0,3);
	\end{circuitikz}
\end{center}
}

\note{If students are confused by this answer, draw the source on the board, then a box around it. Shade in the box so the voltage source is obscured, and now the problem is identical to having a black box voltage source. This particular style of question has featured on several exams, and it's good to have lots of practice with this when calculating power.
}

\sol{
\begin{center}
	\begin{circuitikz}[scale=0.75]
		\draw 
		(0,0) to[V, v=$1\si{\volt}$, invert] ++(0,3)
		(0,0) to[open] ++(.5,0)
			to[open, v_=$V_\text{element}$] ++(0,3)
		(0,0) to[open, i=$I_\text{element}$, invert] ++(0,3);
	\end{circuitikz}
\end{center}
}

\item{ 
\begin{center}
	\begin{circuitikz}[scale=0.75]
		\draw 
		(0,0) to[V, v=$V_\text{S}$, i=$I_\text{element}$, invert] ++(0,3);
	\end{circuitikz}
\end{center}
}
\sol{
\begin{center}
	\begin{circuitikz}[scale=0.75]
		\draw 
		(0,0) to[V, v=$V_\text{S}$, i=$I_\text{element}$, invert] ++(0,3)
			to[open] ++(.5,0)
			to[open, v^=$V_\text{element}$] ++(0,-3);
	\end{circuitikz}
\end{center}
}

\newpage
\item{
\begin{center}
	\begin{circuitikz}[scale=0.75]
		\draw 
		(0,0) to[V, v=$-1\si{\volt}$, invert] ++(0,3)
		(0,0) to[open, i=$I_\text{element}$, invert] ++(0,3);
	\end{circuitikz}
\end{center}
}
\sol{
\begin{center}
	\begin{circuitikz}[scale=0.75]
		\draw 
		(0,0) to[V, v=$-1\si{\volt}$, invert] ++(0,3)
		(0,0) to[open] ++(.5,0)
			to[open, v_=$V_\text{element}$] ++(0,3)
		(0,0) to[open, i=$I_\text{element}$, invert] ++(0,3);
	\end{circuitikz}
\end{center}
}

\item{\textbf{(PRACTICE)}

\begin{center}
	\begin{circuitikz}[scale=0.75]
		\draw 
		(0,0) to[I, l=$I_\text{S}$, invert] ++(0,3)
			to[open] ++(.5,0)
			to[open, v^=$V_\text{element}$] ++(0,-3);
	\end{circuitikz}
\end{center}
}
\sol{
\begin{center}
	\begin{circuitikz}[scale=0.75]
		\draw 
		(0,0) to[I, l=$I_\text{S}$, invert] ++(0,3)
			to[open] ++(.5,0)
			to[open, v^=$V_\text{element}$] ++(0,-3)
		(0,3) to[open, i^=$I_\text{element}$] ++(0,-3);
	\end{circuitikz}
\end{center}
}

\item{\textbf{(PRACTICE)}

\begin{center}
	\begin{circuitikz}[scale=0.75]
		\draw 
		(0,0) to[I, l=$I_\text{S}$, invert] ++(0,3)
		(0,0) to[open] ++(.5,0)
			to[open, v_=$V_\text{element}$] ++(0,3);
	\end{circuitikz}
\end{center}
}
\sol{
\begin{center}
	\begin{circuitikz}[scale=0.75]
		\draw 
		(0,0) to[I, l=$I_\text{S}$, invert] ++(0,3)			
		(.5,0) to[open, v_=$V_\text{element}$] ++(0,3)
		(0,0) to[open, i^=$I_\text{element}$] ++(0,3);
	\end{circuitikz}
\end{center}
}

\item{\textbf{(PRACTICE)}

\begin{center}
	\begin{circuitikz}[scale=0.75]
		\draw 
		(0,0) to[I, l=$I_\text{S}$, invert] ++(0,3)
			to[open, i=$I_\text{element}$] ++(0,-3);
	\end{circuitikz}
\end{center}
}
\sol{
\begin{center}
	\begin{circuitikz}[scale=0.75]
		\draw 
		(0,0) to[I, l=$I_\text{S}$, invert] ++(0,3)
			to[open, i=$I_\text{element}$] ++(0,-3)
		(.5,3) to[open, v^=$V_\text{element}$] ++(0,-3);
	\end{circuitikz}
\end{center}
}

\item{\textbf{(PRACTICE)}

\begin{center}
	\begin{circuitikz}[scale=0.75]
		\draw 
		(0,0) to[I, l=$I_\text{S}$, invert] ++(0,3)
		(0,0) to[open, i=$I_\text{element}$, invert] ++(0,3);
	\end{circuitikz}
\end{center}
}
\sol{
\begin{center}
	\begin{circuitikz}[scale=0.75]
		\draw 
		(0,0) to[I, l=$I_\text{S}$, invert] ++(0,3)
		(0,0) to[open] ++(.5,0)
			to[open, v_=$V_\text{element}$] ++(0,3)
		(0,0) to[open, i=$I_\text{element}$, invert] ++(0,3);
	\end{circuitikz}
\end{center}
}

\item{\textbf{(PRACTICE)}

\begin{center}
	\begin{circuitikz}[scale=0.75]
		\draw (0,0) to[R, i=$I_\text{element}$] ++(-4,0);
	\end{circuitikz}
\end{center}
}
\sol{
\begin{center}
	\begin{circuitikz}[scale=0.75]
		\draw (0,0) to[R, v=$V_\text{element}$, i=$I_\text{element}$] ++(-4,0);
	\end{circuitikz}
\end{center}
}

\item{\textbf{(PRACTICE)}

\begin{center}
	\begin{circuitikz}[scale=0.75]
		\draw (0,0) to[R, v=$V_\text{element}$] ++(-4,0);
	\end{circuitikz}
\end{center}
}
\sol{
\begin{center}
	\begin{circuitikz}[scale=0.75]
		\draw (0,0) to[R, v=$V_\text{element}$, i=$I_\text{element}$] ++(-4,0);
	\end{circuitikz}
\end{center}
}
\end{enumerate}